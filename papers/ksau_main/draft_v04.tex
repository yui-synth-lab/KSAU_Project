\section{Topological Mass Formula from Hyperbolic 3-Manifold Invariants:
A Data-Driven Framework with 69 Hypothesis
Tests}\label{topological-mass-formula-from-hyperbolic-3-manifold-invariants-a-data-driven-framework-with-69-hypothesis-tests}

\textbf{Authors:} Yuya Yamamoto \textbf{Date:} 2026-03-01
\textbf{Version:} draft\_v4 \textbf{Status:} UNDER REVIEW

\begin{center}\rule{0.5\linewidth}{0.5pt}\end{center}

\subsection{Abstract}\label{abstract}

We present a unified topological framework for Standard Model (SM)
particle masses and mixing parameters based on hyperbolic 3-manifold
invariants. By mapping 12 SM particles to unique link topologies, we
demonstrate a fermion mass formula \(\ln(m) = \eta \kappa V_{eff} + C\)
with \(R^2 = 0.9998\), where \(\kappa = \pi/24\) is derived from 24-cell
resonance conditions. The framework successfully derives the
gravitational constant \(G\) with a precision of \(0.000026\%\) and
reproduces the Jarlskog invariant and signs of the CKM matrix with
\(R^2 = 0.9980\). This work summarizes the results of 69 hypothesis
tests conducted across 26 AIRDP cycles, including 24 documented
rejections which delineate the statistical boundaries of the theory.
Hypothesis generation and iterative testing were conducted via the AIRDP
framework, with all statistical criteria pre-registered before each
cycle.

\begin{center}\rule{0.5\linewidth}{0.5pt}\end{center}

\subsection{1. Introduction}\label{introduction}

The relationship between the geometry of spacetime and the properties of
elementary particles remains one of the most profound questions in
theoretical physics. While topological quantum field theories (TQFT)
have provided deep insights into knot invariants, their direct
application to the Standard Model mass hierarchy has been limited. This
paper introduces the KSAU (Knot/String/Anyon Unified) framework, which
identifies particles with specific hyperbolic 3-manifolds and uses their
invariants to explain physical observables. Unlike traditional top-down
models, KSAU utilizes the AI Research Development Protocol (AIRDP) to
systematically test correlations against real data. Our contribution is
the rigorous documentation of 69 hypothesis tests, establishing a
data-driven foundation for topological correlates of the SM mass
spectrum. The framework's origin lies in a geometric observation:
non-trivial knots exist only in three spatial dimensions, making
3-manifold invariants natural candidates for encoding particle
properties.

\begin{center}\rule{0.5\linewidth}{0.5pt}\end{center}

\subsection{2. Theoretical Framework}\label{theoretical-framework}

The KSAU framework posits that elementary particles are encoded as
hyperbolic 3-manifolds. The fundamental mass scale is governed by the
24-cell resonance constant:
\[\kappa = \frac{\pi}{24} \approx 0.1308997\] This constant acts as the
slope in the mass-volume scaling law. The fermion mass formula is
defined as: \[\ln(m) = \eta \kappa V_{eff} + C\] where
\(V_{eff} = V + a \cdot n + b \cdot \ln(Det) + c\) is the effective
volume, incorporating corrections for crossing number \(n\) and
determinant \(Det\). The sector-specific coefficients \(\eta\) are
derived from the projection geometry of a 10D bulk onto a 9D boundary
(H65), yielding \(\eta = 20.0\) for leptons and \(\eta = 10.0\) for
quarks. The global intercept \(C\) is linked to the boson mass scale,
theoretically derived as:
\[C = \pi\sqrt{3} + \frac{1}{10} \approx 5.5414\] The \(1/10\) term
arises from the projection of the 10D bulk volume onto the boundary
manifold (H68), ensuring consistency with the observed boson mass
hierarchy.

\begin{center}\rule{0.5\linewidth}{0.5pt}\end{center}

\subsection{3. Methods}\label{methods}

Verification was conducted via the AIRDP cycles, ensuring statistical
independence and rigor.

\subsubsection{3.1 Statistical Protocol}\label{statistical-protocol}

Every hypothesis was tested using the following criteria: -
\textbf{Bonferroni Correction:} Significance threshold
\(\alpha = 0.05 / 3 \approx 0.0167\) per cycle. - \textbf{Monte Carlo
Validation:} \(n=10,000\) trials with \texttt{seed=42} to determine the
False Positive Rate (FPR). - \textbf{LOO-CV:} Leave-One-Out
Cross-Validation for all regression models to ensure generalizability. -
\textbf{SSoT Adherence:} All constants were sourced from the Single
Source of Truth (\texttt{ssot/constants.json}).

The discretization constant \(K=24\) is not treated as a free parameter
or independently tested hypothesis; it is a mathematical consequence of
the 24-cell resonance condition \(K(4)\cdot\kappa=\pi\) (H6). Treating
\(K=24\) as a candidate for statistical validation would constitute a
circular test (H23: FPR=93.82\%); therefore, it is fixed as a
theoretical constant throughout all analyses.

\subsubsection{3.2 Data Sources}\label{data-sources}

We utilized the KnotInfo and LinkInfo databases, extracting volume
(\(V\)), crossing number (\(n\)), determinant (\(Det\)), signature
(\(s\)), and unknotting number (\(u\)) for 6,502 knots and links up to
12 crossings. All primary results are based exclusively on real
observational data.

\begin{center}\rule{0.5\linewidth}{0.5pt}\end{center}

\subsection{4. Results}\label{results}

\subsubsection{4.1 Fermion Mass Formula}\label{fermion-mass-formula}

The regression of 9 fermions against their assigned topological
\(V_{eff}\) achieved \(R^2 = 0.9998\).

\begin{figure}
\centering
\pandocbounded{\includegraphics[keepaspectratio,alt={Fig. 1: Fermion Mass Formula}]{figures/fig_01_mass_formula.png}}
\caption{Fig. 1: Fermion Mass Formula}
\end{figure}

{\def\LTcaptype{} % do not increment counter
\begin{longtable}[]{@{}llll@{}}
\toprule\noalign{}
Model & Free Parameters & Observations & Ratio \\
\midrule\noalign{}
\endhead
\bottomrule\noalign{}
\endlastfoot
Mass Formula & 4 (\(a, b, c, C\)) & 9 & 4:9 \\
\end{longtable}
}

\subsubsection{4.2 Topological Assignment
Uniqueness}\label{topological-assignment-uniqueness}

The mapping of 12 particles to unique knots was verified using a
permutation test. The current assignment achieves a 12/12 match with
\(p = 0.0\) and \(FPR = 0.0\) (0/10,000 permutations; resolution =
\(10^{-4}\)), indicating that the mapping is statistically unique.

\begin{figure}
\centering
\pandocbounded{\includegraphics[keepaspectratio,alt={Fig. 3: Topology Assignment Uniqueness}]{figures/fig_03_uniqueness.png}}
\caption{Fig. 3: Topology Assignment Uniqueness}
\end{figure}

\subsubsection{4.3 Gravitational Constant
G}\label{gravitational-constant-g}

The Newton gravitational constant \(G\) was derived from the 10D bulk
volume and 24-cell compactification (H53):
\[G_{derived} = 6.708001762 \times 10^{-39} \text{ GeV}^{-2}\] This
represents a deviation of only \(0.0000263\%\) from the experimental
value \(6.708 \times 10^{-39} \text{ GeV}^{-2}\). This derivation uses
zero free parameters.

\subsubsection{4.4 CKM Matrix and Quantum
Numbers}\label{ckm-matrix-and-quantum-numbers}

The CKM matrix magnitudes and signs were reproduced with
\(R^2 = 0.9980\) using a logit-geometric model with Jones polynomial
features (H67).

\begin{figure}
\centering
\pandocbounded{\includegraphics[keepaspectratio,alt={Fig. 4: CKM Matrix Comparison}]{figures/fig_04_ckm_matrix.png}}
\caption{Fig. 4: CKM Matrix Comparison}
\end{figure}

{\def\LTcaptype{} % do not increment counter
\begin{longtable}[]{@{}llll@{}}
\toprule\noalign{}
Model & Free Parameters & Observations & Ratio \\
\midrule\noalign{}
\endhead
\bottomrule\noalign{}
\endlastfoot
CKM Model & 5 (\(A, B, \beta, \gamma, C\)) & 9 & 5:9 \\
\end{longtable}
}

Quantum numbers \((Q, S, G)\) were successfully mapped to writhe and
signature invariants with \(100\%\) accuracy and \(FPR=0.0\) (H66) with
zero free parameters.

\subsubsection{4.5 Dark Matter Candidates (optional
prediction)}\label{dark-matter-candidates-optional-prediction}

Using rule-based criteria (\(\det \equiv 0 \pmod{24}\) and TSI \(\geq\)
24), we identified 67 link topologies as stable dark matter candidates
(H30, Cycle 12). However, it should be noted that the negative
correlation observed in H60 (OR=0.745) suggests a potential theoretical
conflict with this stability index, and no experimental verification
pathway currently exists.

\subsubsection{4.6 Hypothesis Testing
Overview}\label{hypothesis-testing-overview}

Over 26 cycles, 69 hypotheses were tested, resulting in 41 Accepted, 24
Rejected, and 4 Modified outcomes.

\begin{figure}
\centering
\pandocbounded{\includegraphics[keepaspectratio,alt={Fig. 2: Hypothesis Test Scorecard}]{figures/fig_02_scorecard.png}}
\caption{Fig. 2: Hypothesis Test Scorecard}
\end{figure}

\begin{center}\rule{0.5\linewidth}{0.5pt}\end{center}

\subsection{5. Discussion}\label{discussion}

\subsubsection{5.1 Successes and
Limitations}\label{successes-and-limitations}

While the fermion mass formula shows exceptional correlation, the
framework faces significant quantitative limits. CKM transitions that
are Cabibbo-forbidden (\(u \to b, t \to d\)) show errors between
\(63\%\) and \(100\%\), indicating that the current geometric model
captures the hierarchy but lacks fine-grained precision for suppressed
transitions. Similarly, PMNS angles achieve an MSE of
\(5.44 \text{ deg}^2\), which is only qualitative agreement.

\subsubsection{5.2 Negative Results}\label{negative-results}

We report several critical rejections that define the theory's
boundaries: - \textbf{H33/H47:} The constant \(\kappa\) cannot be
independently recovered from mass regression; it is a fundamental input,
not a derived parameter. - \textbf{H58:} A joint test of axion mass,
gravity deviation, and Top width failed to reach significance
(\(p=0.067\)). Notably, the specific axion mass prediction
\(m_a = 12.16\ \mu\text{eV}\) falls within the ADMX 2023 exploration
range (11--14 μeV). While the individual prediction is statistically
motivated, the failure of the joint test suggests that individual
successes do not yet aggregate into a unified predictive power. -
\textbf{H59:} Torsion-based mass corrections showed high training
\(R^2\) but failed LOO-CV (\(R^2_{LOO}=0.11\)), indicating overfitting.
- \textbf{H60:} The predicted correlation between
\(Det \equiv 0 \pmod{24}\) and stability was found to be negative
(\(OR=0.745\)), contradicting early 24-cell symmetry assumptions.

\begin{figure}
\centering
\pandocbounded{\includegraphics[keepaspectratio,alt={Fig. 5: Negative Results Summary}]{figures/fig_05_negative_results.png}}
\caption{Fig. 5: Negative Results Summary}
\end{figure}

\begin{center}\rule{0.5\linewidth}{0.5pt}\end{center}

\subsection{6. Conclusion}\label{conclusion}

The KSAU framework establishes a statistically significant connection
between 3-manifold invariants and the Standard Model spectrum. We have
found statistically significant correlations for mass hierarchy
(\(R^2=0.9998\)), gravity (\(0.0000263\%\)), and quantum numbers. While
significant challenges remain in predicting mixing angles and suppressed
transitions, the 69 tests documented here provide a robust, transparent,
and reproducible map of the topological phase space of the SM spectrum.

\begin{center}\rule{0.5\linewidth}{0.5pt}\end{center}

\subsection{References}\label{references}

\begin{enumerate}
\def\labelenumi{\arabic{enumi}.}
\tightlist
\item
  Witten, E. (1989). ``Quantum Field Theory and the Jones Polynomial.''
  \emph{Comm. Math. Phys.} 121:351-399.
\item
  Atiyah, M. (1990). ``The Geometry and Physics of Knots.''
  \emph{Lezioni Lincee}, Cambridge University Press.
\item
  KnotInfo Database (2026). ``Table of Knot Invariants.''
  \url{https://knotinfo.math.indiana.edu/}
\item
  Eto, M., Hamada, Y. \& Nitta, M. (2025). ``Tying Knots in Particle
  Physics.'' \emph{Phys. Rev.~Lett.} 135, 091603.
\item
  KSAU Project Team. (2026). ``AIRDP Project Status Report: Cycle
  01-26.'' (Zenodo Archive, pending DOI).
\end{enumerate}

\begin{center}\rule{0.5\linewidth}{0.5pt}\end{center}

\subsection{Appendix A: Full Hypothesis Index
(H1--H69)}\label{appendix-a-full-hypothesis-index-h1h69}

Appendix A contains all 69 hypotheses as historical records. Note: H2
was initially validated against synthetic ln\_ST targets; real-data
validation is recorded in H12 (Cycle 6). H7 R²=0.528 is based on the GPR
model verdict (Cycle 4).

{\def\LTcaptype{} % do not increment counter
\begin{longtable}[]{@{}
  >{\raggedright\arraybackslash}p{(\linewidth - 8\tabcolsep) * \real{0.0909}}
  >{\raggedright\arraybackslash}p{(\linewidth - 8\tabcolsep) * \real{0.3864}}
  >{\raggedright\arraybackslash}p{(\linewidth - 8\tabcolsep) * \real{0.1591}}
  >{\raggedright\arraybackslash}p{(\linewidth - 8\tabcolsep) * \real{0.1818}}
  >{\raggedright\arraybackslash}p{(\linewidth - 8\tabcolsep) * \real{0.1818}}@{}}
\toprule\noalign{}
\begin{minipage}[b]{\linewidth}\raggedright
ID
\end{minipage} & \begin{minipage}[b]{\linewidth}\raggedright
Hypothesis Name
\end{minipage} & \begin{minipage}[b]{\linewidth}\raggedright
Cycle
\end{minipage} & \begin{minipage}[b]{\linewidth}\raggedright
Status
\end{minipage} & \begin{minipage}[b]{\linewidth}\raggedright
Metric
\end{minipage} \\
\midrule\noalign{}
\endhead
\bottomrule\noalign{}
\endlastfoot
H1 & Statistical Hypothesis Innocence Proof & 1 & accepted & R²=0.999 \\
H2 & Axion ST Uncertainty & 1 & accepted* & R²=0.767* \\
H3 & CS Mapping & 1 & rejected & - \\
H4 & Axion-ST Correlation & 2 & rejected & p=0.0588 \\
H5 & CS-V Mapping k2 & 2 & accepted & FPR=0.0166 \\
H6 & κ Geometric Derivation & 3 & accepted & error=0\% \\
H7 & ST Refinement GPR & 4 & accepted & R²=0.528 \\
H8 & CS Mapping Redesign & 4 & rejected & - \\
H9 & ST Scaling & 5 & modified & - \\
H10 & k-Function Integrity & 5 & rejected & - \\
H11 & V=0→V Phase Transition & 5 & accepted & R²=0.9995 \\
H12 & Axion ST Real Data & 6 & accepted & R²=0.519 \\
H13 & WRT TQFT Mapping & 6 & rejected & - \\
H14 & Axion Uncertainty GPR & 7 & accepted & - \\
H15 & Algebraic CS Mapping & 7 & accepted & - \\
H16 & κ Geometric Derivation v2 & 8 & accepted & - \\
H17 & Lifetime Correlation & 8 & accepted & - \\
H18 & Phase Viscosity & 8 & modified & - \\
H19 & Phase Viscosity Correction & 9 & accepted & - \\
H20 & G Derivation & 9 & accepted & error\textless1e-6 \\
H21 & DM Prediction & 9 & rejected & FPR=72.66\% \\
H22 & κ from Resonance & 10 & rejected & p=0.0354 \\
H23 & Phase Discretization & 10 & rejected & FPR=93.82\% \\
H24 & TSI Lifetime & 10 & accepted & R²=0.9129 \\
H25 & Deterministic Quantization & 11 & accepted & - \\
H26 & TSI Universal & 11 & rejected & p=0.1734 \\
H27 & DM Candidates Top 10 & 11 & accepted & - \\
H28 & Decay Width TSI & 12 & rejected & p=0.4310 \\
H29 & ST Mass Correction & 12 & rejected & p=0.0588 \\
H30 & DM Parity Constraint & 12 & accepted & - \\
H31 & Decay Width Multi-regression & 13 & accepted & R²=0.8015 \\
H32 & ST Torsion & 13 & rejected & p=0.0712 \\
H33 & κ Independent Recovery & 13 & rejected & - \\
H34 & Linear ST Correction & 14 & rejected & p=0.0712 \\
H35 & V\_eff Recovery & 14 & accepted & - \\
H36 & κ First-Principles & 14 & accepted & - \\
H37 & Decay Width Topological & 15 & rejected & p=0.1610 \\
H38 & ST Lepton Correction & 15 & rejected & p=0.25 \\
H39 & 24-cell Resonance Derivation & 16 & accepted & - \\
H40 & Holistic V\_eff Validation & 16 & rejected & p=0.0970 \\
H41 & Lepton Mass Inversion & 17 & accepted & error=1.72\% \\
H42 & Boson Shift Derivation & 17 & accepted & - \\
H43 & Refined TSI & 17 & accepted & - \\
H44 & κ `24' Derivation & 18 & accepted & - \\
H45 & Linear ST All Fermions & 18 & rejected & p=0.22 \\
H46 & 10D Gravity Precision & 19 & accepted & - \\
H47 & κ Regression via V\_eff & 19 & rejected & - \\
H48 & Non-linear ET Correction & 19 & rejected & p=0.0435 \\
H49 & First-Principles Assignment & 20 & accepted & 12/12 match \\
H50 & Novel Predictions & 20 & accepted & - \\
H51 & SM Gauge D4 Embedding & 20 & accepted & - \\
H52 & τ-V Correlation & 21 & rejected & p=0.0213 \\
H53 & 24-cell Compactification G & 21 & accepted & 0.0000263\% \\
H54 & Mathematical Rigor (Σ=m/M\_P) & 21 & accepted & - \\
H55 & 24-cell Assignment Rule & 22 & accepted & - \\
H56 & Novel Predictions MC & 22 & modified & - \\
H57 & ST+LOO Stability & 22 & modified & - \\
H58 & Joint MC Predictions & 23 & rejected & p=0.067 \\
H59 & ST LOO Stability & 23 & rejected & LOO-R²=0.11 \\
H60 & Det ≡ 0 (mod 24) Stability & 23 & rejected & OR=0.7452 \\
H61 & Real Topological Stability Index Derivation & 24 & accepted & - \\
H62 & Torsion Correction LOO Stability Verification & 24 & accepted &
- \\
H63 & Mathematical Rigor and Dimensional Analysis Integration & 24 &
accepted & - \\
H64 & Brunnian Stability Rule Uniqueness Proof & 25 & accepted & FPR =
0.0 \\
H65 & Quark Coefficient (10.0) First-Principles Derivation & 25 &
accepted & error\textless0.04\% \\
H66 & Quantum Numbers Geometric Origin & 25 & accepted & 100\% sign \\
H67 & CKM Matrix Geometric Origin & 26 & accepted & R²=0.9980 \\
H68 & Boson Mass Intercept C Geometric Derivation & 26 & accepted &
error \textless{} 1e-12 \\
H69 & Early Universe Topology Selection Mechanism & 26 & accepted &
p=0.0022 \\
\end{longtable}
}

*Note: H2 was validated against synthetic ln\_ST data.

\begin{center}\rule{0.5\linewidth}{0.5pt}\end{center}

\subsection{Revision Notes (Revision
4)}\label{revision-notes-revision-4}

\begin{itemize}
\tightlist
\item
  \textbf{{[}R-1{]} LaTeX Regression Fix}: Restored all backslashes
  (\texttt{\textbackslash{}}) in LaTeX commands that were erroneously
  replaced by tab characters in v03. Specifically fixed
  \texttt{\textbackslash{}times}, \texttt{\textbackslash{}text},
  \texttt{\textbackslash{}to}, and
  \texttt{\textbackslash{}mu\textbackslash{}text\{eV\}} in Sections 4.3,
  5.1, and 5.2. Verified literal backslash retention in the markdown
  source.
\item
  \textbf{{[}R-2{]} DM Candidate Caveat (H60 Reference)}: Added a
  cross-reference to H60 in Section 4.5, noting the potential
  theoretical conflict arising from the observed negative correlation
  between the \(Det \equiv 0 \pmod{24}\) rule and the TSI stability
  index.
\item
  \textbf{{[}R-3{]} H7 R² Source}: Added a clarifying note in Appendix A
  regarding the origin of the \(R^2=0.528\) value for H7 from the GPR
  model verdict.
\item
  \textbf{{[}Misc{]} Consistency Updates}: Unified gravity precision
  notation to ``0.0000263\%'' in the conclusion to match Section 4.3.
\end{itemize}
